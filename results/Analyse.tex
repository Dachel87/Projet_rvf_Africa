% Options for packages loaded elsewhere
\PassOptionsToPackage{unicode}{hyperref}
\PassOptionsToPackage{hyphens}{url}
%
\documentclass[
]{article}
\usepackage{amsmath,amssymb}
\usepackage{iftex}
\ifPDFTeX
  \usepackage[T1]{fontenc}
  \usepackage[utf8]{inputenc}
  \usepackage{textcomp} % provide euro and other symbols
\else % if luatex or xetex
  \usepackage{unicode-math} % this also loads fontspec
  \defaultfontfeatures{Scale=MatchLowercase}
  \defaultfontfeatures[\rmfamily]{Ligatures=TeX,Scale=1}
\fi
\usepackage{lmodern}
\ifPDFTeX\else
  % xetex/luatex font selection
\fi
% Use upquote if available, for straight quotes in verbatim environments
\IfFileExists{upquote.sty}{\usepackage{upquote}}{}
\IfFileExists{microtype.sty}{% use microtype if available
  \usepackage[]{microtype}
  \UseMicrotypeSet[protrusion]{basicmath} % disable protrusion for tt fonts
}{}
\makeatletter
\@ifundefined{KOMAClassName}{% if non-KOMA class
  \IfFileExists{parskip.sty}{%
    \usepackage{parskip}
  }{% else
    \setlength{\parindent}{0pt}
    \setlength{\parskip}{6pt plus 2pt minus 1pt}}
}{% if KOMA class
  \KOMAoptions{parskip=half}}
\makeatother
\usepackage{xcolor}
\usepackage[margin=1in]{geometry}
\usepackage{color}
\usepackage{fancyvrb}
\newcommand{\VerbBar}{|}
\newcommand{\VERB}{\Verb[commandchars=\\\{\}]}
\DefineVerbatimEnvironment{Highlighting}{Verbatim}{commandchars=\\\{\}}
% Add ',fontsize=\small' for more characters per line
\usepackage{framed}
\definecolor{shadecolor}{RGB}{248,248,248}
\newenvironment{Shaded}{\begin{snugshade}}{\end{snugshade}}
\newcommand{\AlertTok}[1]{\textcolor[rgb]{0.94,0.16,0.16}{#1}}
\newcommand{\AnnotationTok}[1]{\textcolor[rgb]{0.56,0.35,0.01}{\textbf{\textit{#1}}}}
\newcommand{\AttributeTok}[1]{\textcolor[rgb]{0.13,0.29,0.53}{#1}}
\newcommand{\BaseNTok}[1]{\textcolor[rgb]{0.00,0.00,0.81}{#1}}
\newcommand{\BuiltInTok}[1]{#1}
\newcommand{\CharTok}[1]{\textcolor[rgb]{0.31,0.60,0.02}{#1}}
\newcommand{\CommentTok}[1]{\textcolor[rgb]{0.56,0.35,0.01}{\textit{#1}}}
\newcommand{\CommentVarTok}[1]{\textcolor[rgb]{0.56,0.35,0.01}{\textbf{\textit{#1}}}}
\newcommand{\ConstantTok}[1]{\textcolor[rgb]{0.56,0.35,0.01}{#1}}
\newcommand{\ControlFlowTok}[1]{\textcolor[rgb]{0.13,0.29,0.53}{\textbf{#1}}}
\newcommand{\DataTypeTok}[1]{\textcolor[rgb]{0.13,0.29,0.53}{#1}}
\newcommand{\DecValTok}[1]{\textcolor[rgb]{0.00,0.00,0.81}{#1}}
\newcommand{\DocumentationTok}[1]{\textcolor[rgb]{0.56,0.35,0.01}{\textbf{\textit{#1}}}}
\newcommand{\ErrorTok}[1]{\textcolor[rgb]{0.64,0.00,0.00}{\textbf{#1}}}
\newcommand{\ExtensionTok}[1]{#1}
\newcommand{\FloatTok}[1]{\textcolor[rgb]{0.00,0.00,0.81}{#1}}
\newcommand{\FunctionTok}[1]{\textcolor[rgb]{0.13,0.29,0.53}{\textbf{#1}}}
\newcommand{\ImportTok}[1]{#1}
\newcommand{\InformationTok}[1]{\textcolor[rgb]{0.56,0.35,0.01}{\textbf{\textit{#1}}}}
\newcommand{\KeywordTok}[1]{\textcolor[rgb]{0.13,0.29,0.53}{\textbf{#1}}}
\newcommand{\NormalTok}[1]{#1}
\newcommand{\OperatorTok}[1]{\textcolor[rgb]{0.81,0.36,0.00}{\textbf{#1}}}
\newcommand{\OtherTok}[1]{\textcolor[rgb]{0.56,0.35,0.01}{#1}}
\newcommand{\PreprocessorTok}[1]{\textcolor[rgb]{0.56,0.35,0.01}{\textit{#1}}}
\newcommand{\RegionMarkerTok}[1]{#1}
\newcommand{\SpecialCharTok}[1]{\textcolor[rgb]{0.81,0.36,0.00}{\textbf{#1}}}
\newcommand{\SpecialStringTok}[1]{\textcolor[rgb]{0.31,0.60,0.02}{#1}}
\newcommand{\StringTok}[1]{\textcolor[rgb]{0.31,0.60,0.02}{#1}}
\newcommand{\VariableTok}[1]{\textcolor[rgb]{0.00,0.00,0.00}{#1}}
\newcommand{\VerbatimStringTok}[1]{\textcolor[rgb]{0.31,0.60,0.02}{#1}}
\newcommand{\WarningTok}[1]{\textcolor[rgb]{0.56,0.35,0.01}{\textbf{\textit{#1}}}}
\usepackage{graphicx}
\makeatletter
\def\maxwidth{\ifdim\Gin@nat@width>\linewidth\linewidth\else\Gin@nat@width\fi}
\def\maxheight{\ifdim\Gin@nat@height>\textheight\textheight\else\Gin@nat@height\fi}
\makeatother
% Scale images if necessary, so that they will not overflow the page
% margins by default, and it is still possible to overwrite the defaults
% using explicit options in \includegraphics[width, height, ...]{}
\setkeys{Gin}{width=\maxwidth,height=\maxheight,keepaspectratio}
% Set default figure placement to htbp
\makeatletter
\def\fps@figure{htbp}
\makeatother
\setlength{\emergencystretch}{3em} % prevent overfull lines
\providecommand{\tightlist}{%
  \setlength{\itemsep}{0pt}\setlength{\parskip}{0pt}}
\setcounter{secnumdepth}{-\maxdimen} % remove section numbering
\ifLuaTeX
  \usepackage{selnolig}  % disable illegal ligatures
\fi
\usepackage{bookmark}
\IfFileExists{xurl.sty}{\usepackage{xurl}}{} % add URL line breaks if available
\urlstyle{same}
\hypersetup{
  pdftitle={R Notebook},
  hidelinks,
  pdfcreator={LaTeX via pandoc}}

\title{R Notebook}
\author{}
\date{\vspace{-2.5em}}

\begin{document}
\maketitle

Charger les bibliothèques nécessaires

\begin{Shaded}
\begin{Highlighting}[]
\FunctionTok{library}\NormalTok{(dplyr)}
\end{Highlighting}
\end{Shaded}

\begin{verbatim}
## 
## Attachement du package : 'dplyr'
\end{verbatim}

\begin{verbatim}
## Les objets suivants sont masqués depuis 'package:stats':
## 
##     filter, lag
\end{verbatim}

\begin{verbatim}
## Les objets suivants sont masqués depuis 'package:base':
## 
##     intersect, setdiff, setequal, union
\end{verbatim}

\begin{Shaded}
\begin{Highlighting}[]
\FunctionTok{library}\NormalTok{(readr)}
\FunctionTok{library}\NormalTok{(stringr)}
\end{Highlighting}
\end{Shaded}

Definir le repertoire de travail

\begin{Shaded}
\begin{Highlighting}[]
\FunctionTok{setwd}\NormalTok{(}\StringTok{"/home/lnsp/Bureau/training\_ghana/Travaux\_pratique/rvf\_africa\_projet/donnees"}\NormalTok{)  }\CommentTok{\# Remplace par ton chemin réel}

\NormalTok{fichier }\OtherTok{\textless{}{-}} \StringTok{"rvf\_africa.tsv"}

\NormalTok{donnees }\OtherTok{\textless{}{-}} \FunctionTok{read\_tsv}\NormalTok{(fichier, }\AttributeTok{show\_col\_types =} \ConstantTok{FALSE}\NormalTok{)}

\FunctionTok{print}\NormalTok{(}\FunctionTok{colnames}\NormalTok{(donnees))}
\end{Highlighting}
\end{Shaded}

\begin{verbatim}
##  [1] "Species"            "GenomeStatus"       "Strain"            
##  [4] "Segment"            "GenBank Accessions" "Size"              
##  [7] "GC Content"         "Contig N50"         "CollectionDate"    
## [10] "CollectionYear"     "IsolationCountry"   "GeographicGroup"   
## [13] "HostName"           "Host Common Name"   "HostGroup"
\end{verbatim}

Étape 1 : Retirer les entrées sans date de collecte ou pays

\begin{Shaded}
\begin{Highlighting}[]
\NormalTok{donnees\_filtrees }\OtherTok{\textless{}{-}}\NormalTok{ donnees }\SpecialCharTok{\%\textgreater{}\%}
  \FunctionTok{filter}\NormalTok{(}\SpecialCharTok{!}\FunctionTok{is.na}\NormalTok{(CollectionDate), }\SpecialCharTok{!}\FunctionTok{is.na}\NormalTok{(IsolationCountry))}

\CommentTok{\# Sauvegarder les données filtrées}
\FunctionTok{write\_tsv}\NormalTok{(donnees\_filtrees, }\StringTok{"donnees\_filtrees.tsv"}\NormalTok{)}
\end{Highlighting}
\end{Shaded}

Étape 2 : Grouper et résumer le nombre de séquences par pays, année,
segment

\begin{Shaded}
\begin{Highlighting}[]
\NormalTok{resume\_pays\_annee\_segment }\OtherTok{\textless{}{-}}\NormalTok{ donnees\_filtrees }\SpecialCharTok{\%\textgreater{}\%}
  \FunctionTok{group\_by}\NormalTok{(IsolationCountry, CollectionYear, Segment) }\SpecialCharTok{\%\textgreater{}\%}
  \FunctionTok{summarise}\NormalTok{(}\AttributeTok{Nb\_sequences =} \FunctionTok{n}\NormalTok{(), }\AttributeTok{.groups =} \StringTok{"drop"}\NormalTok{)}

\FunctionTok{write\_tsv}\NormalTok{(resume\_pays\_annee\_segment, }\StringTok{"resume\_pays\_annee\_segment.tsv"}\NormalTok{)}
\FunctionTok{print}\NormalTok{(resume\_pays\_annee\_segment)}
\end{Highlighting}
\end{Shaded}

\begin{verbatim}
## # A tibble: 246 x 4
##    IsolationCountry CollectionYear Segment Nb_sequences
##    <chr>                     <dbl> <chr>          <int>
##  1 Angola                     1985 M                  1
##  2 Angola                     1985 S                  1
##  3 Angola                     2016 L                  1
##  4 Angola                     2016 M                  1
##  5 Angola                     2016 S                  1
##  6 Burkina Faso               1983 L                  1
##  7 Burkina Faso               1983 M                  1
##  8 Burkina Faso               1983 S                  2
##  9 Burundi                    2022 L                  5
## 10 Burundi                    2022 M                  7
## # i 236 more rows
\end{verbatim}

Étape 3 : Créer une colonne dérivée (par exemple : région depuis le
pays)

\begin{Shaded}
\begin{Highlighting}[]
\NormalTok{donnees\_mutation }\OtherTok{\textless{}{-}}\NormalTok{ donnees\_filtrees }\SpecialCharTok{\%\textgreater{}\%}
  \FunctionTok{mutate}\NormalTok{(}\AttributeTok{Region =} \FunctionTok{case\_when}\NormalTok{(}
\NormalTok{    IsolationCountry }\SpecialCharTok{\%in\%} \FunctionTok{c}\NormalTok{(}\StringTok{"Kenya"}\NormalTok{, }\StringTok{"Sudan"}\NormalTok{, }\StringTok{"Tanzania"}\NormalTok{, }\StringTok{"Uganda"}\NormalTok{) }\SpecialCharTok{\textasciitilde{}} \StringTok{"Afrique de l\textquotesingle{}Est"}\NormalTok{,}
\NormalTok{    IsolationCountry }\SpecialCharTok{\%in\%} \FunctionTok{c}\NormalTok{(}\StringTok{"Senegal"}\NormalTok{, }\StringTok{"Mauritania"}\NormalTok{) }\SpecialCharTok{\textasciitilde{}} \StringTok{"Afrique de l\textquotesingle{}Ouest"}\NormalTok{,}
    \ConstantTok{TRUE} \SpecialCharTok{\textasciitilde{}} \StringTok{"Autre"}
\NormalTok{  ))}
\end{Highlighting}
\end{Shaded}

Etape 4: compter le nombre de sequences par hote commun

\begin{Shaded}
\begin{Highlighting}[]
\NormalTok{compte\_hotes }\OtherTok{\textless{}{-}}\NormalTok{ donnees\_mutation }\SpecialCharTok{\%\textgreater{}\%}
  \FunctionTok{group\_by}\NormalTok{(}\StringTok{\textasciigrave{}}\AttributeTok{Host Common Name}\StringTok{\textasciigrave{}}\NormalTok{) }\SpecialCharTok{\%\textgreater{}\%}
  \FunctionTok{summarise}\NormalTok{(}\AttributeTok{Nb\_sequences =} \FunctionTok{n}\NormalTok{(), }\AttributeTok{.groups =} \StringTok{"drop"}\NormalTok{) }\SpecialCharTok{\%\textgreater{}\%}
  \FunctionTok{arrange}\NormalTok{(}\FunctionTok{desc}\NormalTok{(Nb\_sequences))}

\FunctionTok{write\_tsv}\NormalTok{(compte\_hotes, }\StringTok{"compte\_hotes.tsv"}\NormalTok{)}
\FunctionTok{print}\NormalTok{(compte\_hotes)}
\end{Highlighting}
\end{Shaded}

\begin{verbatim}
## # A tibble: 16 x 2
##    `Host Common Name`      Nb_sequences
##    <chr>                          <int>
##  1 Human                            769
##  2 Cow                              243
##  3 Mosquito                         131
##  4 <NA>                             130
##  5 Sheep                             70
##  6 Goat                              18
##  7 Buffalo                            9
##  8 Bat                                7
##  9 Null                               6
## 10 Cattle                             5
## 11 Camel                              3
## 12 Mouse                              3
## 13 Tick                               3
## 14 Lab host                           1
## 15 Northern House Mosquito            1
## 16 Sand fly                           1
\end{verbatim}

Étape 5 : Filtrer par pays, années et segments spécifiques (exemple :
Kenya, 2007, segment L)

\begin{Shaded}
\begin{Highlighting}[]
\NormalTok{filtres\_specifiques }\OtherTok{\textless{}{-}}\NormalTok{ donnees\_mutation }\SpecialCharTok{\%\textgreater{}\%}
  \FunctionTok{filter}\NormalTok{(IsolationCountry }\SpecialCharTok{==} \StringTok{"Kenya"}\NormalTok{, CollectionYear }\SpecialCharTok{==} \DecValTok{2007}\NormalTok{, Segment }\SpecialCharTok{==} \StringTok{"L"}\NormalTok{)}

\FunctionTok{print}\NormalTok{(filtres\_specifiques)}
\end{Highlighting}
\end{Shaded}

\begin{verbatim}
## # A tibble: 44 x 16
##    Species   GenomeStatus Strain Segment `GenBank Accessions`  Size `GC Content`
##    <chr>     <chr>        <chr>  <chr>   <chr>                <dbl>        <dbl>
##  1 Phlebovi~ Partial      K2     L       PP746417              6386         43.8
##  2 Phlebovi~ Partial      J8     L       PP746426              6251         43.6
##  3 Phlebovi~ Partial      J10    L       PP746425              6232         43.5
##  4 Phlebovi~ Partial      J9     L       PP746427              6277         43.5
##  5 Phlebovi~ Partial      KEM-JC L       PP746437              6268         43.5
##  6 Phlebovi~ Partial      KEM-BR L       PP746436              6232         43.5
##  7 Phlebovi~ Partial      KEM-ND L       PP746438              6232         43.4
##  8 Phlebovi~ Partial      MO-LE  L       PP746448              6387         43.5
##  9 Phlebovi~ Partial      HA-HAR L       PP746447              6373         43.9
## 10 Phlebovi~ Partial      MSA    L       PP746449              6388         43.4
## # i 34 more rows
## # i 9 more variables: `Contig N50` <dbl>, CollectionDate <chr>,
## #   CollectionYear <dbl>, IsolationCountry <chr>, GeographicGroup <chr>,
## #   HostName <chr>, `Host Common Name` <chr>, HostGroup <chr>, Region <chr>
\end{verbatim}

Étape 6 : Renommer la colonne CollectionYear en year

\begin{Shaded}
\begin{Highlighting}[]
\NormalTok{donnees\_renommee }\OtherTok{\textless{}{-}}\NormalTok{ donnees\_mutation }\SpecialCharTok{\%\textgreater{}\%}
  \FunctionTok{rename}\NormalTok{(}\AttributeTok{year =}\NormalTok{ CollectionYear)}
\end{Highlighting}
\end{Shaded}

Étape 7 : Grouper et résumer par pays, année et segment

\begin{Shaded}
\begin{Highlighting}[]
\NormalTok{resume\_final }\OtherTok{\textless{}{-}}\NormalTok{ donnees\_renommee }\SpecialCharTok{\%\textgreater{}\%}
  \FunctionTok{group\_by}\NormalTok{(IsolationCountry, year, Segment) }\SpecialCharTok{\%\textgreater{}\%}
  \FunctionTok{summarise}\NormalTok{(}\AttributeTok{Nb\_sequences =} \FunctionTok{n}\NormalTok{(), }\AttributeTok{.groups =} \StringTok{"drop"}\NormalTok{)}

\FunctionTok{write\_tsv}\NormalTok{(resume\_final, }\StringTok{"resume\_final.tsv"}\NormalTok{)}
\FunctionTok{print}\NormalTok{(resume\_final)}
\end{Highlighting}
\end{Shaded}

\begin{verbatim}
## # A tibble: 246 x 4
##    IsolationCountry  year Segment Nb_sequences
##    <chr>            <dbl> <chr>          <int>
##  1 Angola            1985 M                  1
##  2 Angola            1985 S                  1
##  3 Angola            2016 L                  1
##  4 Angola            2016 M                  1
##  5 Angola            2016 S                  1
##  6 Burkina Faso      1983 L                  1
##  7 Burkina Faso      1983 M                  1
##  8 Burkina Faso      1983 S                  2
##  9 Burundi           2022 L                  5
## 10 Burundi           2022 M                  7
## # i 236 more rows
\end{verbatim}

Étape 8 : Identifier les pays avec le plus grand nombre de séquences
complètes (indépendamment du segment)

\begin{Shaded}
\begin{Highlighting}[]
\NormalTok{pays\_plus\_sequences }\OtherTok{\textless{}{-}}\NormalTok{ donnees\_filtrees }\SpecialCharTok{\%\textgreater{}\%}
  \FunctionTok{filter}\NormalTok{(GenomeStatus }\SpecialCharTok{==} \StringTok{"Complete"}\NormalTok{) }\SpecialCharTok{\%\textgreater{}\%}
  \FunctionTok{group\_by}\NormalTok{(IsolationCountry) }\SpecialCharTok{\%\textgreater{}\%}
  \FunctionTok{summarise}\NormalTok{(}\AttributeTok{Nb\_sequences\_completes =} \FunctionTok{n}\NormalTok{(), }\AttributeTok{.groups =} \StringTok{"drop"}\NormalTok{) }\SpecialCharTok{\%\textgreater{}\%}
  \FunctionTok{arrange}\NormalTok{(}\FunctionTok{desc}\NormalTok{(Nb\_sequences\_completes))}

\FunctionTok{write\_tsv}\NormalTok{(pays\_plus\_sequences, }\StringTok{"pays\_plus\_sequences.tsv"}\NormalTok{)}
\FunctionTok{print}\NormalTok{(pays\_plus\_sequences)}
\end{Highlighting}
\end{Shaded}

\begin{verbatim}
## # A tibble: 19 x 2
##    IsolationCountry         Nb_sequences_completes
##    <chr>                                     <int>
##  1 South Africa                                389
##  2 Kenya                                       178
##  3 Madagascar                                   38
##  4 Zimbabwe                                     35
##  5 Central African Republic                     33
##  6 Egypt                                        33
##  7 Uganda                                       28
##  8 Sudan                                        22
##  9 Mauritania                                   15
## 10 Namibia                                      12
## 11 Tanzania                                     12
## 12 Guinea                                        7
## 13 Senegal                                       7
## 14 Angola                                        4
## 15 Burkina Faso                                  4
## 16 Gabon                                         4
## 17 Mayotte                                       2
## 18 Somalia                                       1
## 19 Zambia                                        1
\end{verbatim}

Étape 9 : Compter les hôtes les plus fréquents (top 10)

\begin{Shaded}
\begin{Highlighting}[]
\NormalTok{top\_hotes }\OtherTok{\textless{}{-}}\NormalTok{ donnees\_filtrees }\SpecialCharTok{\%\textgreater{}\%}
  \FunctionTok{group\_by}\NormalTok{(}\StringTok{\textasciigrave{}}\AttributeTok{Host Common Name}\StringTok{\textasciigrave{}}\NormalTok{) }\SpecialCharTok{\%\textgreater{}\%}
  \FunctionTok{summarise}\NormalTok{(}\AttributeTok{Nb\_sequences =} \FunctionTok{n}\NormalTok{(), }\AttributeTok{.groups =} \StringTok{"drop"}\NormalTok{) }\SpecialCharTok{\%\textgreater{}\%}
  \FunctionTok{arrange}\NormalTok{(}\FunctionTok{desc}\NormalTok{(Nb\_sequences)) }\SpecialCharTok{\%\textgreater{}\%}
  \FunctionTok{slice\_head}\NormalTok{(}\AttributeTok{n =} \DecValTok{10}\NormalTok{)}

\FunctionTok{write\_tsv}\NormalTok{(top\_hotes, }\StringTok{"top\_hotes.tsv"}\NormalTok{)}
\FunctionTok{print}\NormalTok{(top\_hotes)}
\end{Highlighting}
\end{Shaded}

\begin{verbatim}
## # A tibble: 10 x 2
##    `Host Common Name` Nb_sequences
##    <chr>                     <int>
##  1 Human                       769
##  2 Cow                         243
##  3 Mosquito                    131
##  4 <NA>                        130
##  5 Sheep                        70
##  6 Goat                         18
##  7 Buffalo                       9
##  8 Bat                           7
##  9 Null                          6
## 10 Cattle                        5
\end{verbatim}

\begin{Shaded}
\begin{Highlighting}[]
\FunctionTok{cat}\NormalTok{(}\StringTok{"✅ Analyse terminée. Tous les fichiers résumés ont été sauvegardés dans le répertoire : "}\NormalTok{, }\FunctionTok{getwd}\NormalTok{(), }\StringTok{"}\SpecialCharTok{\textbackslash{}n}\StringTok{"}\NormalTok{)}
\end{Highlighting}
\end{Shaded}

\begin{verbatim}
## ✅ Analyse terminée. Tous les fichiers résumés ont été sauvegardés dans le répertoire :  /home/lnsp/Bureau/training_ghana/Travaux_pratique/rvf_africa_projet/resultats
\end{verbatim}

Telechargement de librairie

\begin{Shaded}
\begin{Highlighting}[]
\FunctionTok{library}\NormalTok{(dplyr)}
\FunctionTok{library}\NormalTok{(readr)}
\CommentTok{\# repertoire de travail}

\FunctionTok{setwd}\NormalTok{(}\StringTok{"/home/lnsp/Bureau/training\_ghana/Travaux\_pratique/rvf\_africa\_projet/resultats"}\NormalTok{)}

\NormalTok{donnees }\OtherTok{\textless{}{-}} \FunctionTok{read\_tsv}\NormalTok{(fichier, }\AttributeTok{show\_col\_types =} \ConstantTok{FALSE}\NormalTok{)}

\FunctionTok{print}\NormalTok{(}\FunctionTok{colnames}\NormalTok{(donnees))}
\end{Highlighting}
\end{Shaded}

\begin{verbatim}
##  [1] "Species"            "GenomeStatus"       "Strain"            
##  [4] "Segment"            "GenBank Accessions" "Size"              
##  [7] "GC Content"         "Contig N50"         "CollectionDate"    
## [10] "CollectionYear"     "IsolationCountry"   "GeographicGroup"   
## [13] "HostName"           "Host Common Name"   "HostGroup"
\end{verbatim}

1️⃣ Nombre d'isolats par pays

\begin{Shaded}
\begin{Highlighting}[]
\NormalTok{isolats\_par\_pays }\OtherTok{\textless{}{-}}\NormalTok{ donnees }\SpecialCharTok{\%\textgreater{}\%}
  \FunctionTok{group\_by}\NormalTok{(IsolationCountry) }\SpecialCharTok{\%\textgreater{}\%}
  \FunctionTok{summarise}\NormalTok{(}\AttributeTok{Nombre\_isolats =} \FunctionTok{n}\NormalTok{(), }\AttributeTok{.groups =} \StringTok{"drop"}\NormalTok{) }\SpecialCharTok{\%\textgreater{}\%}
  \FunctionTok{arrange}\NormalTok{(}\FunctionTok{desc}\NormalTok{(Nombre\_isolats))}
\FunctionTok{print}\NormalTok{(}\StringTok{"=== Nombre d\textquotesingle{}isolats par pays ==="}\NormalTok{)}
\end{Highlighting}
\end{Shaded}

\begin{verbatim}
## [1] "=== Nombre d'isolats par pays ==="
\end{verbatim}

\begin{Shaded}
\begin{Highlighting}[]
\FunctionTok{print}\NormalTok{(isolats\_par\_pays)}
\end{Highlighting}
\end{Shaded}

\begin{verbatim}
## # A tibble: 25 x 2
##    IsolationCountry         Nombre_isolats
##    <chr>                             <int>
##  1 South Africa                        434
##  2 Kenya                               260
##  3 Madagascar                          198
##  4 Mauritania                          118
##  5 Uganda                              111
##  6 Senegal                              52
##  7 Egypt                                51
##  8 Zimbabwe                             51
##  9 Central African Republic             37
## 10 Sudan                                23
## # i 15 more rows
\end{verbatim}

Sauvegarder le résumé

\begin{Shaded}
\begin{Highlighting}[]
\FunctionTok{write\_tsv}\NormalTok{(isolats\_par\_pays, }\StringTok{"nombre\_isolats\_par\_pays.tsv"}\NormalTok{)}
\end{Highlighting}
\end{Shaded}

2️⃣ Nombre d'isolats par an

\begin{Shaded}
\begin{Highlighting}[]
\NormalTok{isolats\_par\_an }\OtherTok{\textless{}{-}}\NormalTok{ donnees }\SpecialCharTok{\%\textgreater{}\%}
  \FunctionTok{group\_by}\NormalTok{(CollectionYear) }\SpecialCharTok{\%\textgreater{}\%}
  \FunctionTok{summarise}\NormalTok{(}\AttributeTok{Nombre\_isolats =} \FunctionTok{n}\NormalTok{(), }\AttributeTok{.groups =} \StringTok{"drop"}\NormalTok{) }\SpecialCharTok{\%\textgreater{}\%}
  \FunctionTok{arrange}\NormalTok{(CollectionYear)}
\FunctionTok{print}\NormalTok{(}\StringTok{"=== Nombre d\textquotesingle{}isolats par an ==="}\NormalTok{)}
\end{Highlighting}
\end{Shaded}

\begin{verbatim}
## [1] "=== Nombre d'isolats par an ==="
\end{verbatim}

\begin{Shaded}
\begin{Highlighting}[]
\FunctionTok{print}\NormalTok{(isolats\_par\_an)}
\end{Highlighting}
\end{Shaded}

\begin{verbatim}
## # A tibble: 54 x 2
##    CollectionYear Nombre_isolats
##             <dbl>          <int>
##  1           1944             10
##  2           1951              8
##  3           1955             24
##  4           1956              4
##  5           1962              5
##  6           1963              2
##  7           1964              1
##  8           1965              1
##  9           1969             11
## 10           1970              5
## # i 44 more rows
\end{verbatim}

\begin{Shaded}
\begin{Highlighting}[]
\FunctionTok{write\_tsv}\NormalTok{(isolats\_par\_an, }\StringTok{"nombre\_isolats\_par\_an.tsv"}\NormalTok{)}
\end{Highlighting}
\end{Shaded}

3️⃣ Répartition des segments (S, M, L) par pays

\begin{Shaded}
\begin{Highlighting}[]
\NormalTok{repartition\_segments }\OtherTok{\textless{}{-}}\NormalTok{ donnees }\SpecialCharTok{\%\textgreater{}\%}
  \FunctionTok{group\_by}\NormalTok{(IsolationCountry, Segment) }\SpecialCharTok{\%\textgreater{}\%}
  \FunctionTok{summarise}\NormalTok{(}\AttributeTok{Nombre =} \FunctionTok{n}\NormalTok{(), }\AttributeTok{.groups =} \StringTok{"drop"}\NormalTok{) }\SpecialCharTok{\%\textgreater{}\%}
  \FunctionTok{arrange}\NormalTok{(IsolationCountry, Segment)}

\FunctionTok{print}\NormalTok{(}\StringTok{"=== Répartition des segments par pays ==="}\NormalTok{)}
\end{Highlighting}
\end{Shaded}

\begin{verbatim}
## [1] "=== Répartition des segments par pays ==="
\end{verbatim}

\begin{Shaded}
\begin{Highlighting}[]
\FunctionTok{print}\NormalTok{(repartition\_segments)}
\end{Highlighting}
\end{Shaded}

\begin{verbatim}
## # A tibble: 67 x 3
##    IsolationCountry         Segment Nombre
##    <chr>                    <chr>    <int>
##  1 Angola                   L            1
##  2 Angola                   M            2
##  3 Angola                   S            2
##  4 Burkina Faso             L            1
##  5 Burkina Faso             M            1
##  6 Burkina Faso             S            2
##  7 Burundi                  L            5
##  8 Burundi                  M            7
##  9 Burundi                  <NA>         5
## 10 Central African Republic L           11
## # i 57 more rows
\end{verbatim}

\begin{Shaded}
\begin{Highlighting}[]
\CommentTok{\# Sauvegarder le résumé}
\FunctionTok{write\_tsv}\NormalTok{(repartition\_segments, }\StringTok{"repartition\_segments\_par\_pays.tsv"}\NormalTok{)}

\FunctionTok{cat}\NormalTok{(}\StringTok{"✅ Analyse descriptive terminée. Résultats sauvegardés dans le dossier : "}\NormalTok{,}
    \FunctionTok{getwd}\NormalTok{(), }\StringTok{"}\SpecialCharTok{\textbackslash{}n}\StringTok{"}\NormalTok{)}
\end{Highlighting}
\end{Shaded}

\begin{verbatim}
## ✅ Analyse descriptive terminée. Résultats sauvegardés dans le dossier :  /home/lnsp/Bureau/training_ghana/Travaux_pratique/rvf_africa_projet/resultats
\end{verbatim}

Assurez-vous que les colonnes sont bien de type approprié

\begin{Shaded}
\begin{Highlighting}[]
\NormalTok{donnees }\OtherTok{\textless{}{-}}\NormalTok{ donnees }\SpecialCharTok{\%\textgreater{}\%}
  \FunctionTok{mutate}\NormalTok{(}\AttributeTok{CollectionYear =} \FunctionTok{as.integer}\NormalTok{(CollectionYear),}
         \AttributeTok{CollectionDate =} \FunctionTok{as.Date}\NormalTok{(CollectionDate, }\AttributeTok{format =} \StringTok{"\%Y{-}\%m{-}\%d"}\NormalTok{))}
\end{Highlighting}
\end{Shaded}

Visualisation des données

1️⃣ Boxplot : distribution des dates de collecte par pays

\begin{Shaded}
\begin{Highlighting}[]
\FunctionTok{library}\NormalTok{(ggplot2)}

\FunctionTok{ggplot}\NormalTok{(donnees, }\FunctionTok{aes}\NormalTok{(}\AttributeTok{x =}\NormalTok{ IsolationCountry, }\AttributeTok{y =}\NormalTok{ CollectionYear)) }\SpecialCharTok{+}
  \FunctionTok{geom\_boxplot}\NormalTok{(}\AttributeTok{fill =} \StringTok{"lightblue"}\NormalTok{) }\SpecialCharTok{+}
  \FunctionTok{labs}\NormalTok{(}\AttributeTok{title =} \StringTok{"Distribution des années de collecte par pays"}\NormalTok{,}
       \AttributeTok{x =} \StringTok{"Pays"}\NormalTok{,}
       \AttributeTok{y =} \StringTok{"Année de collecte"}\NormalTok{) }\SpecialCharTok{+}
  \FunctionTok{theme\_minimal}\NormalTok{() }\SpecialCharTok{+}
  \FunctionTok{theme}\NormalTok{(}\AttributeTok{axis.text.x =} \FunctionTok{element\_text}\NormalTok{(}\AttributeTok{angle =} \DecValTok{45}\NormalTok{, }\AttributeTok{hjust =} \DecValTok{1}\NormalTok{))}
\end{Highlighting}
\end{Shaded}

\begin{verbatim}
## Warning: Removed 35 rows containing non-finite outside the scale range
## (`stat_boxplot()`).
\end{verbatim}

\includegraphics{Analyse_files/figure-latex/unnamed-chunk-18-1.pdf}

\begin{Shaded}
\begin{Highlighting}[]
\FunctionTok{ggsave}\NormalTok{(}\StringTok{"boxplot\_annee\_par\_pays.png"}\NormalTok{, }\AttributeTok{width =} \DecValTok{8}\NormalTok{, }\AttributeTok{height =} \DecValTok{5}\NormalTok{)}
\end{Highlighting}
\end{Shaded}

\begin{verbatim}
## Warning: Removed 35 rows containing non-finite outside the scale range
## (`stat_boxplot()`).
\end{verbatim}

2️⃣ Boxplot : distribution des années de collecte par segment

\begin{Shaded}
\begin{Highlighting}[]
\FunctionTok{ggplot}\NormalTok{(donnees, }\FunctionTok{aes}\NormalTok{(}\AttributeTok{x =}\NormalTok{ Segment, }\AttributeTok{y =}\NormalTok{ CollectionYear)) }\SpecialCharTok{+}
  \FunctionTok{geom\_boxplot}\NormalTok{(}\AttributeTok{fill =} \StringTok{"lightgreen"}\NormalTok{) }\SpecialCharTok{+}
  \FunctionTok{labs}\NormalTok{(}\AttributeTok{title =} \StringTok{"Distribution des années de collecte par segment"}\NormalTok{,}
       \AttributeTok{x =} \StringTok{"Segment viral"}\NormalTok{,}
       \AttributeTok{y =} \StringTok{"Année de collecte"}\NormalTok{) }\SpecialCharTok{+}
  \FunctionTok{theme\_minimal}\NormalTok{()}
\end{Highlighting}
\end{Shaded}

\begin{verbatim}
## Warning: Removed 35 rows containing non-finite outside the scale range
## (`stat_boxplot()`).
\end{verbatim}

\includegraphics{Analyse_files/figure-latex/unnamed-chunk-19-1.pdf}

\begin{Shaded}
\begin{Highlighting}[]
\FunctionTok{ggsave}\NormalTok{(}\StringTok{"boxplot\_annee\_par\_segment.png"}\NormalTok{, }\AttributeTok{width =} \DecValTok{6}\NormalTok{, }\AttributeTok{height =} \DecValTok{5}\NormalTok{)}
\end{Highlighting}
\end{Shaded}

\begin{verbatim}
## Warning: Removed 35 rows containing non-finite outside the scale range
## (`stat_boxplot()`).
\end{verbatim}

3️⃣ Résumé : nombre d'isolats par pays et année

\begin{Shaded}
\begin{Highlighting}[]
\NormalTok{resume\_pays\_annee }\OtherTok{\textless{}{-}}\NormalTok{ donnees }\SpecialCharTok{\%\textgreater{}\%}
  \FunctionTok{group\_by}\NormalTok{(IsolationCountry, CollectionYear) }\SpecialCharTok{\%\textgreater{}\%}
  \FunctionTok{summarise}\NormalTok{(}\AttributeTok{Nb\_isolats =} \FunctionTok{n}\NormalTok{(), }\AttributeTok{.groups =} \StringTok{"drop"}\NormalTok{) }\SpecialCharTok{\%\textgreater{}\%}
  \FunctionTok{arrange}\NormalTok{(}\FunctionTok{desc}\NormalTok{(Nb\_isolats))}

\FunctionTok{print}\NormalTok{(}\StringTok{"Résumé des isolats par pays et année :"}\NormalTok{)}
\end{Highlighting}
\end{Shaded}

\begin{verbatim}
## [1] "Résumé des isolats par pays et année :"
\end{verbatim}

\begin{Shaded}
\begin{Highlighting}[]
\FunctionTok{print}\NormalTok{(resume\_pays\_annee)}
\end{Highlighting}
\end{Shaded}

\begin{verbatim}
## # A tibble: 113 x 3
##    IsolationCountry CollectionYear Nb_isolats
##    <chr>                     <int>      <int>
##  1 South Africa               2010        279
##  2 Madagascar                 2008        171
##  3 Kenya                      2007        152
##  4 Uganda                     2018         49
##  5 South Africa               2008         35
##  6 Kenya                      2006         27
##  7 Mauritania                 1987         27
##  8 South Africa               2011         27
##  9 Zimbabwe                   1978         27
## 10 South Africa               2009         26
## # i 103 more rows
\end{verbatim}

\begin{Shaded}
\begin{Highlighting}[]
\FunctionTok{write\_tsv}\NormalTok{(resume\_pays\_annee, }\StringTok{"resume\_isolats\_par\_pays\_annee.tsv"}\NormalTok{)}
\end{Highlighting}
\end{Shaded}

4️⃣ Barplot : distribution des souches par pays

\begin{Shaded}
\begin{Highlighting}[]
\FunctionTok{ggplot}\NormalTok{(donnees, }\FunctionTok{aes}\NormalTok{(}\AttributeTok{x =}\NormalTok{ IsolationCountry, }\AttributeTok{fill =}\NormalTok{ Strain)) }\SpecialCharTok{+}
  \FunctionTok{geom\_bar}\NormalTok{(}\AttributeTok{position =} \StringTok{"dodge"}\NormalTok{) }\SpecialCharTok{+}
  \FunctionTok{labs}\NormalTok{(}\AttributeTok{title =} \StringTok{"Distribution des souches par pays"}\NormalTok{,}
       \AttributeTok{x =} \StringTok{"Pays"}\NormalTok{,}
       \AttributeTok{y =} \StringTok{"Nombre d’isolats"}\NormalTok{,}
       \AttributeTok{fill =} \StringTok{"Souche (Strain)"}\NormalTok{) }\SpecialCharTok{+}
  \FunctionTok{theme\_minimal}\NormalTok{() }\SpecialCharTok{+}
  \FunctionTok{theme}\NormalTok{(}\AttributeTok{axis.text.x =} \FunctionTok{element\_text}\NormalTok{(}\AttributeTok{angle =} \DecValTok{45}\NormalTok{, }\AttributeTok{hjust =} \DecValTok{1}\NormalTok{))}
\end{Highlighting}
\end{Shaded}

\includegraphics{Analyse_files/figure-latex/unnamed-chunk-21-1.pdf}

\begin{Shaded}
\begin{Highlighting}[]
\FunctionTok{ggsave}\NormalTok{(}\StringTok{"barplot\_souches\_par\_pays.png"}\NormalTok{, }\AttributeTok{width =} \DecValTok{8}\NormalTok{, }\AttributeTok{height =} \DecValTok{5}\NormalTok{)}

\FunctionTok{cat}\NormalTok{(}\StringTok{"✅ Graphiques générés et sauvegardés dans : "}\NormalTok{, }\FunctionTok{getwd}\NormalTok{(), }\StringTok{"}\SpecialCharTok{\textbackslash{}n}\StringTok{"}\NormalTok{)}
\end{Highlighting}
\end{Shaded}

\begin{verbatim}
## ✅ Graphiques générés et sauvegardés dans :  /home/lnsp/Bureau/training_ghana/Travaux_pratique/rvf_africa_projet/resultats
\end{verbatim}

\end{document}
